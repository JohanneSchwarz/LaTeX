\chapter{Einleitung}
Im Umfeld der Finanzbranche und in der DZ BANK AG werden täglich enorme Datenmengen produziert. Besonders durch den Umstand, dass die DZ BANK AG alle Finanzdaten der Volksbanken Reiffeisenbanken abwickelt muss ein System gefunden werden, um alle Daten historisch und inhaltlich richtig abzulegen. Da ebenfalls eine sogenannte Tagesendverarbeitung durchgeführt wird, um eine klare Historisierung aufzubauen, müssen die Daten so abgelegt sein, dass dies geschehen kann. Zu diesem Zwecke werden die Daten in einem (financial) Data Warehouse abgelegt. Dieses ist von der Konzeption auf den Aufbau einer Datenhistorie ausgelegt. 
