\documentclass[a4paper,12pt]{report}

\usepackage[T1]{fontenc}
\usepackage{microtype}
\usepackage[a4paper,top=2.5cm,bottom=2.5cm,left=2.5cm,right=2.5cm]{geometry}
\usepackage[utf8]{inputenc}
%\usepackage{subfigures}
\usepackage{lmodern}
\usepackage[german]{babel}
\usepackage{graphicx}
\usepackage{hyperref}
\usepackage{listings}
\usepackage{float}
\usepackage{chngcntr}
\usepackage{tabularx}\counterwithout{footnote}{chapter}
\usepackage[backend=biber]{biblatex}
\usepackage{pdfpages}
\usepackage{microtype}
\usepackage[bottom]{footmisc}
\usepackage[labelfont={bf,sf},font={small},labelsep=space]{caption}

\lstset{basicstyle=\ttfamily,captionpos=b,numbers=left, stepnumber=1,frame=single, numberstyle = \tiny,breaklines=true}

\title{Aufbau und Bewirtschaftung eines Financial-Data Warehouses (FDW) mit Hilfe des ETL-Tools Informatica}
\author{Johannes Schwarz}
\addbibresource{Literatur.bib}


\begin{document}
\begin{titlepage}
	\centering
	\includegraphics[width=0.25\textwidth]{Images/DHBWLogo.png}\par\vspace{0.25cm}
	{\scshape\LARGE Duale Hochschule Baden-Württemberg Mannheim \par}
	{\scshape Fakultät Technik - Studiengang Angewandte Informatik\par}
	\vspace{1cm}
	{\scshape\Large Praxisarbeit zum 3. Semester\par}
	{Im Zeitraum Januar 2021 bis März 2021\par}
	\vspace{1cm}
	{\huge\bfseries  Aufbau und Bewirtschaftung eines Financial-Data Warehouses (FDW) mit Hilfe des ETL-Tools Informatica\par}
	\vspace{0.5cm}
	{\Large\itshape Johannes Schwarz\par}
	\vspace{0.25cm}
	{\itshape Matrikelnummer, Kurs: 9446662, MA-TINF19AI2\par}
	\vspace{0.5cm}
	{bei der DZ BANK AG, Frankfurt a.M.\par}
	{Betriebliche Betreuer\par}
	{Thomas \textsc{Kerber}\par}
	{\includegraphics[scale=0.2]{Anmerkungen/Unterschriften.png}\par}

	\vfill

% Bottom of the page
	{\large Abgabedatum: 30.09.2021\par}
\end{titlepage}
\tableofcontents
\addcontentsline{end}{chapter}{Literatur}

%\includepdf[pages=1]{Anmerkungen/Reflexion.pdf}

\chapter*{Sperrvermerk}
Die vorliegende Praxisarbeit mit dem Titel:\newline\newline
\glqq Aufbau und Bewirtschaftung eines Financial-Datawarehouses (FDW) mit Hilfe des ETL-Tools Informatica\grqq\newline\newline
beinhaltet interne und vertrauliche Informationen des Unternehmens\newline\newline
DZ BANK AG\newline
Platz der Republik\newline
60265 Frankfurt am Main\newline\newline
Eine Einsicht in diese Praxisarbeit ist nicht gestattet. 
Ausgenommen davon sind die betreuenden Dozenten, sowie die befugten Mitglieder des Prüfungsausschusses. 
Eine Veröffentlichung und Vervielfältigung der Praxisarbeit - auch in Auszügen ist nicht gestattet. 
Ausnahmen von dieser Regelung bedürfen der Genehmigung der DZ BANK AG.
\input{Erklärung}
\chapter*{Abstract}
\section*{Deutsch}
\section*{English}



\chapter{Einleitung}
Im Umfeld der Finanzbranche und in der DZ BANK AG werden täglich enorme Datenmengen produziert. Besonders durch den Umstand, dass die DZ BANK AG alle Finanzdaten der Volksbanken Reiffeisenbanken abwickelt muss ein System gefunden werden, um alle Daten historisch und inhaltlich richtig abzulegen. Da ebenfalls eine sogenannte Tagesendverarbeitung durchgeführt wird, um eine klare Historisierung aufzubauen, müssen die Daten so abgelegt sein, dass dies geschehen kann. Zu diesem Zwecke werden die Daten in einem (financial) Data Warehouse abgelegt. Dieses ist von der Konzeption auf den Aufbau einer Datenhistorie ausgelegt. 

\chapter{Wege der Datenspeicherung}
Es gibt viele verschiedene Wege, um Daten in geordneten Dateien\footnote{Sammlung von Informationen gleichartiger Dinge auf persistentem Speichermedium} abzulegen. Hierfür
können Datenbanken genutzt werden. Eine Datenbank ist eine Sammlung von Daten,
die untereinander in logischer Beziehung stehen“ Es ist eine Organisation von Daten, die
darauf ausgelegt ist auf Dauer persistiert 2 und zusätzlich flexiblen und sicheren Gebrauch
zu gewährleisten. Weiter umfasst die Datenorganisation sowohl eine Datenbasis, als eine
zugehörige Datenverwaltung (oder Datenbankmanagementsystem). Eine Datenbank soll
dazu dienen große Datenmengen strukturiert zu speichern und verwalten.\cite{Huckert:VL1_2}\cite{Gabler_Wirtschaftslex:Datenbank}
\section{Die Operationale Datenbank}
\subsection{Definition}
Eine operationale Datenbank (DB) speichert und verwaltet Daten in Echtzeit. Dies
bedeutet, dass bedeutet, dass Daten sowohl in Echtzeit geschrieben als auch gelöscht
werden können.\cite{techopedia:Operationale_Datenbank} 
Diese Datenbanken können mit der Sprache SQL (Structured query language) editiert werden.
\subsection{Aufbau und Verwendung}
Eine operationale Datenbank speichert Daten innerhalb eines Unternehmens oder einer Organisation. 
Solche Daten können beispielsweise Gehaltsabrechnungen oder auch die Daten von Angestellten sein. \newline 
Die wesentliche Charakteristik einer operationalen Datenbank ist die Auslegung auf die Verarbeitung von Echtzeit-Daten. 
Hier werden Daten bezüglich des Tagesgeschäfts oder Kundendaten gespeichert.
Bei operationalen Datenbanken besteht weiterhin kein anspruch auf eine schnelle Verarbeitung für neue Datensätze, was die Datenbank an sich Unperformater macht.
Da sie allerdings auf beispielsweise ein Tagesgeschäft ausgelegt ist, muss das nicht weiter beachtet werden.
\cite{ComputerWeekly:Datenbanksysteme} \cite{techopedia:Operationale_Datenbank}
\section{Die analytische Datenbank als Gegenstück zu operationalen Datenbank}
Eine Analytische Datenbank ist, im Gegensatz zu einer Operationalen, eine Datenbank, die zeithistorische Daten sammeln kann.
In solchen Datenbanken werden beispielsweise historische Daten von Business-Metriken schreibgeschützt gesammelt. 
Solche Daten können Verkaufszahlen, Lagerbestände oder ähnliches sein.
Verantwortliche des Unternehmens oder Mitarbeiter können aus den Datenbanken Analyse-Daten gewinnen.
Die gespeicherten Daten werden in regelmäßigen Abständen aktualisiert, um die sonstigen Systeme eines Unternehmens in der analytischen Datenbank abzubilden.
Typischerweise ist diese Datenbank darauf ausgelegt eine \glqq Business-Intelligence (BI)\grqq\footnote{\glqq Prozess, der Daten in Informationen transformiert und diese wiederum durch die Anwendung von Erfahrungen in Wissen\grqq\cite{BigData:BI}}~oder andere analytische Anwendungen zu unterstützen, hierfür meist auch als Teil eines Data Warehouses.
Dies unterscheidet die analytische Datenbank von einer Operationalen.
\cite{CW:analytische_Datenbank}\footnote{Es gibt noch weitere Datenbankarten, wie hierachisch-organisierte oder Objektorientierte Datenbanken, auf die ich allerdings im Rahmen dieser Arbeit nicht näher eingehe}
\section{Die Relationale Datenbank als Modellierungsart für Datenbanken}
\subsection{Definition}
\glqq Eine relationale Datenbank ist ein Typ von Datenbanken, der die Speicherung und den Zugriff auf miteinander verbundener Datenpunkte ermöglicht.\grqq\cite{Oracle:rel_DB}\newline
Sie basieren im Grunde auf dem relationalen Modell, welches eine intuitive und einfache Art beschreibt, wie Tabellen darzustellen sind.
\subsubsection{Das relationale Modell}
Das Relationale Modell besteht im wesentlichen aus 3 wichtigen Bausteinen:
\begin{itemize}
	\item {Tabellen}
	\item {Attributen}
	\item {Beziehungen (Relations)}
\end{itemize}
Ein relationales Datenbankmodell ist beschrieben durch eine Ansammlung von Tabellen, die in einer Beziehung (Relation) zueinander stehen. Mathematisch werden hier Relationen als Mengen von Tupeln gesehen. In einer Menge gibt es jedoch keine Duplikate, wohingegen in einer Tabelle, welche mit SQL generiert wurde Duplikate vorhanden sein können.  Jede Zeile einer Tabelle (Tupel) ist demnach ein Datensatz in der Datenbank. Jedes Tupel besitzt eine große Reihe an Eigenschaften (Attribute), die die Spalten einer Tabelle sind. \cite{DatenbankenVerstehen:Relationales_Datenbankmodell}\cite{Huckert:VL4_5}
\begin{figure}[H]
	\begin{center}
		\includegraphics[width=14cm]{Images/Rel_DB_BSP}
		\caption{Beispiel eines relationalen Datenbankmodells \cite{DatenbankenVerstehen:Relationales_Datenbankmodell}}
		\label{fig:ER-Modell}
	\end{center}
\end{figure}
In \autoref{fig:ER-Modell} ist ein Beispiel eines relationalen Datenbankmodells zu sehen.Dieses Modell zeigt die Beziehung verschiedener Tabellen zueinander. Hier wird zudem auch auf das Konzept der Normalisierung geachtet, da ein solches Datenbankmodell nur durch korrekte Normalisierung und deren Normalformen erstellt werden kann (mehr in \autoref{ch:Data_Mod_Normal})
Diese Modellierung ermöglicht es den Entwicklern und Anwendern eine einheitliche Struktur zur Datenspeicherung zu schaffen\cite{Oracle:rel_DB}
\subsection{Aufbau und Verwendung}
\glqq Im relationalen Modell sind die logischen Datenstrukturen - die Datentabellen - von den physischen Datenstrukturen getrennt\grqq\cite{Oracle:rel_DB}\newline
Diese Trennung ermöglicht es den physischen Speicher in Rechenzentren besser Verwalten zu können ohne dass die Struktur der Speicherung von Datensätzen verändert werden muss. Weiter wird eine relationale Datenbank dafür verwendet, um Daten durch Computersysteme zu Speichern und Bearbeiten. \cite{Oracle:rel_DB}\cite{Bigdata-Insider:rel_DB}
Dieses Konzept, Datenbanken nach dem relationalen Modells zu strukturieren, ist von großer Relevanz. 
Dieses Modell bietet viele Vorteile hinsichtlich der Datenhaltung und Datestellung der Datenmodellierung. 
Weiterhin lassen sich dadurch praktisch das Format und die Beziehungen zwischen verschiedenen Entitäten darstellen.
\section{Das Data Warehouse}
\subsection{Definition}
Ein Data Warehouse ist ein zentrales Datenbanksystem, welches hauptsächlich zu Analysezwecken in Unternehmen eingesetzt wird. \glqq Dieses System extrahiert, sammelt und sichert relevante Daten aus verschiedenen heterogenen Datenquellen.\footnote{Eine Datenquelle ist heterogen, wenn die Struktur der Daten uneinheitlich ist.}\grqq \cite{Bigdata-Insider:DW}
\subsection{Aufbau und Verwendung}
Das Data Warehouse (auch Datenlager) ist eine Plattform, die Daten aus verschiedenen Datenquellen sammelt und nachgelagerte Analysesysteme versorgt. Die Datenbestände innerhalb des Data Warehouses sind daher strukturiert, sodass außerhalb des DW eine einheitliche Schnittstelle zum Abfragen der Daten vorhanden ist.
Das Data Warehouseing besteht aus vier verschiedenen Teilprozessen:
\begin{itemize}
	\item {Datenbeschaffung}: Datenextraktion aus verschiedenen unterschiedlichen Datenbeständen
	\item {Datenhaltung}: Speicherung der Daten im Data Warehouse
	\item {Datenversorgung}: Versorgung der Systeme, welche die Daten benötigen, Bereitstellung von Data Marts\footnote{Datenbank, die für die Anforderungen einer bestimmten Benutzergruppe konzipiert ist\cite{Talend:Data_Mart}}
	\item {Datenauswertung}: Analyse und Auswertung der Datenbestände
\end{itemize}
Weiterhin soll das Data Warehouse als die \glqq Single Source of truth\grqq (Die einheitliche Version/Quelle der Wahrheit) dienen und den Benutzern Zugriff auf diese Wahrheit ermöglichen, damit Berichte erstellt oder zeitnahe Geschäftsprognosen getroffen werden können. Das Data Warehouse ist als ein zentrales Informationssystem, in dem Daten aus mehreren Quellen zusammengefasst werden und in dem eine Datenhistorie aufgebaut wird.
\subsection{Zentrale Merkmale des DWH}
\begin{itemize}
	\item Fachorientiert - Das DWH orientier sich an einem bestimmten Thema und liefert dazu Informationen, anstelle der aktuellen Geschäftszahlen
	\item Integriert - Integration von Daten mehrerer Datenquellen und die Vereinheitlichung auf einen genutzten Standard
	\item Zeitunterschied - Die Daten im Data Warehouse liefern Daten in historischer Perspektive
	\item nicht flüchtig - Es gibt kein Löschen von älteren Daten im DWH, sondern es wird eine Historie aufgebaut und das DHW immer erweitert
\end{itemize}
\cite{Bigdata-Insider:DW}\cite{Astera:Kimmball_Inmon}
\section{Der Data Mart}
Der Data Mart ist genauso wie das Data Warehouse ein Repository\footnote{Lager, Depot - verwaltetes Verzeichnis zur Speicherung und Beschreibung digitaler Objekte\cite{Wikipedia:Repository}}, in dem Daten bis zur Verwendung verwaltet und gespeichert werden. Im Gegensatz zum Data Warehouse, in dem sämtliche Informationen gespeichert sind, spiegeln Data Marts nur die Anforderungen einer bestimmten Abteilung oder Geschäftsfunktion. Hauptsächlich sollen Data Marts nur einen kleinen Teildatensatz vom gesamt Data Warehouse isolieren beziehungsweise partitionieren\footnote{Vorgang des Aufteilens}.

Ein Data Mart kann in zwei verschiedenen Ansätzen erstellt werden. Zum einen kann der Top-Down-Ansatz genutzt werden, bei welchem der Data Mart auf der Grundlage eines bereits bestehenden Datawarehouses erstellt werden. Andererseits kann er aber auch aus anderen Quellen, wie internen operativen Systemen oder externen Daten erstellt werden.\cite{Talend:Data_Mart}
\begin{table}[H]
	\centering
	\begin{tabular}{l|ll}
						& Data Mart 			& Data Warehouse \\
						\hline
	Größe				& < 100 GB				& > 100 GB \\
	Subjekt				& Ein Subjekt			& Mehrere Subjekte\\
	Umfang				& Unternehmensbereich	& Gesamtes Unternehmen\\
	Datenquellen		& Wenige Quellen		& Viele Quellsysteme\\
	Datenintegration	& Ein Themenbereich		& Alle Geschäftsdaten
	\end{tabular}
	\caption{Verdeutlichung der Unterschiede zwischen Data Mart und Data Warehouse}
\end{table}
\section{Der Data Lake}
\subsection{Definition}
Der Data Lake ist ein System oder Repository in dem Daten im Rohformatformat\footnote{Schreibe von Daten auf ein Speichermedium, ohne Bearbeitung, häufig auch ohne Digitalisierung} zu speichern.Der Data Lake kann strukturierte\footnote{Daten mit einer erkennbaren Struktur} und nicht strukturierte Daten aufnehmen. Weiterhin lässt sich der Data Lake zur Big-Data-Analyse einsetzen\footnote{Massendatenanalyse, angesichts des Umfangs der Thematik werde ich darauf nicht näher eingehen}.\cite{Wikipedia:Data_Lake} \cite{BigData-Insider:Data_Lake}
\subsection{Aufbau und Verwendung}
Im Data Lake sollte gewisse Grundfunktionalitäten vorhanden sein, um die Anforderungen der Anwendungen zu erfüllen, die auf den Data Lake als Informationsquelle zurückgreifen. Durch die Ablage sowohl von strukturierten, als auch von unstrukturierten Daten sollen Datensilos\footnote{Datenbestände und Informationen, die an verschieden Orten liegen und auf die nur bestimmte Abteilungen oder Nutzergruppen innerhalb des Unternehmens Zugriff haben\cite{DigitalWiki:Datensilo}} vermieden werden. Im Gegensatz zum Data Warehouse können die Daten im Data Lake demnach auch unstrukturiert sein. Dies bedeutet, dass es auch möglich ist beispielsweise Bilder in einem Data Lake abzulegen.  \cite{BigData-Insider:Data_Lake}\cite{AT:Data_Lake}

\chapter{Datenmodellierung und Normalisierung}\label{ch:Data_Mod_Normal}
Die Datenmodellierung spielt in der Gestaltung einer Datenbank eine wichtige und entscheidende Rolle. Sie bildet eine wesentliche Schnittstelle in der Kommunikation mit dem Kunden, oder innerbetrieblich zwischen Fachbereich und IT-Bereich. Hierfür gibt es unterschiedliche Modelle, die die Entwurfsphase einer Datenbank beschreiben:
\begin{itemize}
	\item {konzeptioneller Datenbankentwurf}\newline
		Beim konzeptionellen Datenmodell wird definiert, welche Businessobjekte aus der \glqq realen Welt\grqq~auftauchen und welche Beziehungen diese untereinander haben. 
	\item {Fachlicher Datenbankentwurf}\newline
		Beim Fachlichen Datenbankentwurf geht es um eine Anforderungs- und Informationsanalyse.\cite{HS-Karslruhe:DB_Entwurf}Dieser Entwurf wird mit dem Fachbereich, der die Datenbank nutzen will durchgeführt, um erwartete Werte und potentielle Fehlerquellen identifizieren und mit Constraints\footnote{Einschränkungen, die Fehleingaben verhindern} vermeiden zu können. Weiterhin werden die Kardinalitäten\footnote{Beziehungsverhältnis zwischen Entitätsmengen\cite{Huckert:VL1_2}} zwischen den Entitäten definiert. Allerdings findet hier keine Generalisierung statt. Dies bedeutet, dass keine \glqq Vergröberung der Klassen\grqq, also weniger Informationen zu einer Klasse, vorgenommen wird.\cite{Möller:STS} Auf Grundlage dieses Entwurfes steht die Kommunikation mit dem Kunden, der meist keine fundierten Kenntnisse über die Datenbank besitzt.\cite{Möller:STS}
	\item {Logischer Datenbankentwurf}\newline
		Der nächste Schritt ist ein logischer Datenbankentwurf. Ziel dieses Entwurfs ist die Korrektheit und Lesbarkeit des Datenbankmodells, aber auch die Ablage von Daten in der Datenbank. So wird in dieser Phase beispielsweise festgelegt, ob alle Produkte in einer Tabelle festgehalten werden, oder ob alle Geschäftspartner in einer Tabelle stehen oder getrennt nach Käufer/Verkäufer in verschiedenen. Das Modell kann dabei in 2 verschiedene Teile untergliedert werden:
		\begin{itemize}
			\item {Systemunabhängig} \newline
				Beim systemunabhängigen Modellen wird mit festen Abbildungsregeln\footnote{Gewisse Regeln, die abhängig von gewissen unternehmerischen Faktoren sind} gearbeitet. Dies bedeutet, dass diese Modellierung abhängig vom verwendeten Modell ist. Daraus ergibt sich ein Datenbankmodell, welches nur auf eine Datenbank mit gleichem Modell übertragen werden kann.
			\item {Systemabhängig}  \newline
				Systemabhängige Datenmodelle verwenden spezifische Funktionen und Funktionalitäten einer Datenbank. Daraus ergeben sich ein oder mehrere sogenannte DDL-Skript\footnote{Data Definition Language - Sprache, zur Anlage und Löschung von Datenbanken\cite{Huckert:VL1_2}}. Diese Skripte sind in der Sprache der jeweiligen Datenbank geschrieben und ermöglichen eine bessere Protierung\footnote{(über-)tragen oder Transport} auf andere Datenbanken, die die gleich Sprache sprechen. \newline\cite{DatenbankenVerstehen:logischer_Datenbankentwurf}\cite{DatenbankenVerstehen:konzeptioneller_Datenbankentwurf}
		\end{itemize}
	Weiterhin wird hier festgelegt, welche Modellierungsart (Data Vault, ER-Diagramm (\autoref{fig:ER-Modell}), Star-Modellierung(\autoref{fig:Star-Schema}), etc. )genutzt wird. 
	\item {physischer Datenbankentwurf}\newline
	Zuletzt steht der physische Datenbankentwurf. Hier werden die datenbankspezifischen Komponenten betrachtet. In dieser Phase muss sich auf eine Datenbank festgelegt werden und Aspekte wie:
	\begin{itemize}
		\item {Indizes:} Indexstruktur in einer Datenbank, die die Suche und das Sortieren nach bestimmten Einträgen/Datensätzen beschleunigt\cite{Wikipedia:Datenbankindex}
		\item {Partitionierung:} Vorgang des Aufteilens eines physischen Datenträgers
		\item {Speicherung:} 
		\item {Spezielle Funktionen der Datenbank:} Funktionen, welche je nach Hersteller einer Datenbank unterschiedlich sein können. (Unterschiede zwischen Oracle- oder MSSQL-Datenbanken)
	\end{itemize}
	\cite{ITEW:Patrik_02.05}\cite{DatenbankenVerstehen:Physischer_Datenbankentwurf}
\end{itemize} 
\section{Von der Datenbank zum Data Warehouse}
Das allgemeine Ziel der Datenbanken ist es eine strukturierte Speicherung von Daten zu gewährleisten. Dazu sollen zum einen so wenige Redundanzen\footnote{lat.:Überfluss - Mehrfaches abspeichern von Datensätzen} wie möglich erzeugt werden, zum anderen allerdings auch eine korrekte Eingabe der Daten zu ermöglichen. Dazu gibt es verschiedene Konzepte, die eingesetzt werden, um Datenmodellierung umzusetzen und auch zu standardisieren.
\subsection{Die Theorie der Normalformen}
Die Theorie der Normalformen versucht 3 Punkte zu bewirken:
\begin{enumerate}
	\item {Eliminieren von Redundanzen - Vermeiden mehrmaligen Speicherns des gleichen Datensatzes}
	\item {Eliminieren von Anomalien - Vermeiden von Schwierigkeiten im Zusammenhang mit DML-Optionen\footnote{Data Manipulition Language - Sprache zum einfügen, löschen und ändern von Datensätzen}}
	\item {Festhalten von realitätskonformen Sachverhalten - Gewährleistung der Richtigkeit eines Datensatzes}
\end{enumerate}
Als eine der wichtigsten Normalform in der Datenbankverwaltung hat sich die 3. Normalform etabliert. Diese Normalform schließt alle Eigenschaften der darunter liegenden (1. und 2.) Normalformen mit ein. 
\subsubsection*{1.Normalform}
Die Wertebereiche der Tabelle muss atomar sein, das bedeutet, dass der Wert eines Attributes nicht weiter in kleinere Werte heruntergebrochen werden kann.
\begin{figure}[H]
	\begin{center}
		\includegraphics[scale=0.4]{Images/1NF}
		\caption{Vergleich Atomarer/nicht-Atomarer Wertebereich\cite{Knott:Normalformen}}
		\label{fig:1NF}
	\end{center}
\end{figure}
In \autoref{fig:1NF} ist zusehen, das in der 2. Tabelle besser nach bestimmten Datensätzen zu suchen ist, da keine Doppeltnennungen in einem Attribut vorhanden sind.\newpage

\subsubsection*{2. Normalform}
Zusätzlich muss jedes Nichtschlüsselattribut vom Primärschlüssel\footnote{Attribut, welches eine eindeutige Identifikation der realen Objekte erlaubt\cite{Wikipedia:Primärschlüssel}} voll funktional abhängig sein und nicht bereits von einem Teil der Schlüsselattribute
\begin{figure}[H]
	\begin{center}
		\includegraphics[scale=0.4]{Images/2NF}
		\caption{Tabelle(n) in 2. Normalform\cite{Knott:Normalformen}}
		\label{fig:2NF}
	\end{center}
\end{figure}
Mathematisch gesehen ist ein Attribut B vom Attribut A voll funktional abhängig, wenn B alleine durch A bestimmt wird: $A\rightarrow B$.
In \autoref{fig:1NF} ist das Attribut \colorbox[rgb]{.8,.8,.8}{\textcolor{orange}{Schwerpunkt}} voll funktional Abhängig vom Attribut Kurs, Allerdings hängt der Schwerpunkt eines Kurses nicht nur vom Kurs alleine ab. Diese Funktionale Abhängigkeit versucht die 2. Normalform zu vermeiden und so wird diese Tabelle in zwei getrennte Tabellen ausgelagert.
Wie in \autoref{fig:2NF} zu sehen ist ist nun der Kurs voll funktional Abhängig.

\subsubsection*{3. Normalform}
Zusätzlich zu den Bedingungen der 1. und 2. Normalform darf Nicht-Schlüsselmerkmal transitiv von irgendeinem anderen Schlüssel abhängig sein. Transitiv abhängig bedeutet, dass ein Merkmal über Umwege funktional abhängig ist. Mathematisch kann das wie folgt ausgedrückt werden: $A\rightarrow B\rightarrow C$. Dies bedeutet: $C$ ist von $A$ transitiv abhängig, wenn es voll von $B$ und $B$ voll von $A$ funktional abhängig ist.
\begin{figure}[H]
	\begin{center}
		\includegraphics[scale=0.4]{Images/3NF}
		\caption{Tabelle(n) in 3. Normalform\cite{Knott:Normalformen}}
		\label{fig:3NF}
	\end{center}
\end{figure}
In \autoref{fig:2NF} ist das Attribut \colorbox[rgb]{.8,.8,.8}{\textcolor{orange}{Trakt}} transitiv vom Schlüsselattribut \colorbox[rgb]{.8,.8,.8}{\textcolor{orange}{Kurs}} abhängig: Trakt ist alleine vom Schwerpunkt und der Schwerpunkt alleine vom Kurs abhängig (Trakt$\rightarrow$ Schwerpunkt$\rightarrow$ Kurs).
Damit diese Tabelle gänzlich ohne transitive Abhängigkeiten gespeichert werden kann, müssen Informationen über den Schwerpunkt in eine neue Tabelle extrahiert werden.
\cite{Huckert:VL3}\cite{Knott:Normalformen}
Zwar bietet diese Art der Modellierung einige Vorteile, aber auch ein paar Nachteile, die es zu beachten gilt:
\begin{itemize}
	\item {Vorteile}
	\begin{itemize}
		\item Alle Datensätze werden nur einmal gespeichert
		\item Wenige Redundanzen
	\end{itemize}
	\item{Nachteile}
	\begin{itemize}
		\item Schwer eine Historie zu erstellen (Gerade im Data Warehouse Umfeld wichtig)
		\item Es sind viele Änderungen bei kleinen Datensatzanpassungen nötig
		\item Es kann keine Parallelisierung bei Abfragen oder Einlesen vorgenommen werden
	\end{itemize}
\end{itemize}
\subsection{Data Warehouse nach Kimball und Inmon und Data Vault nach Linstedt}
Die Entwicklung in größeren Unternehmen geht hinsichtlich der Analyse von großen Datenbeständen in die Richtung des Data Warehouses. Zur Datenmodellierung gibt es zwei unterschiedliche Ansätze: Der Ansatz nach Kimball und der Ansatz nach Inmon. Der dritte Ansatz richtet sich nach Linstedt und beschreibt ein Data Vault.

\subsubsection{Data Vault nach Linstedt}\label{subsec:DataVault}
Das Konzept der Data Vault Modellierung wurde in den 1990er-Jahren von Dan Linstedt entwickelt und setzt damit den Fokus auf die Bedürfnisse des Unternehmens.
\begin{quote}
	\glqq Der Data Vault ist ein detailorientierter, historisch nachverfolgender und eindeutig verknüpfter Satz normalisierter Tabellen, die einen oder mehrere Funktionsbereiche des Geschäfts unterstützen. Es handelt sich um einen hybriden Ansatz, der die besten Eigenschaften von 3rd Normal Form (3NF) und Sternschema umfasst. Das Design ist flexibel, skalierbar, konsistent und an die Anforderungen des Unternehmens anpassbar. Es ist ein Datenmodell, das speziell für die Anforderungen von Enterprise Data Warehouses entwickelt wurde.\grqq\cite{DL:Data_Vault_Basics}\cite{DeepL}
\end{quote}
Ein Data Vault ermöglicht demnach eine flexible und Aufwandsarme Anpassung an eine Data Warehouse Lösung.
Ein Data Vault nutzt ein Konzept aus Hubs, Links und Satelliten, um die Skalierung einfacher zu gestalten:
\begin{itemize}
	\item {Hubs}\newline
		Alle Informationen die ein Geschäftskonzept eindeutig beschreiben (bspw.: Kundennummer bei einem Kunden) werden in Hubs abgelegt. Hubs repräsentieren demnach die Kernobjekte der jeweiligen Geschäftslogik. Ein Hub lässt sich als Liste von eindeutigen Geschäftsschlüsseln beschreiben, da er neben einem fachlichen Schlüssel zusätzlich einen künstlichen Primärschlüssel beinhaltet, um fachliche Entitäten zu identifizieren. Weiterhin sind hier technische Informationen über das Quellsystem oder das Ladedatum enthalten, welche als Integrationspunkt für (neue) Daten aus verschiedenen Quellen dienen
	\item {Links}\newline
		In den Links sind alle Beziehungen zwischen Geschäftsobjekten abgelegt. Solch eine Beziehung kann beispielsweise die Zuordnung eines Produktes zu einem Hersteller sein. Links werden demnach als Verknüpfung zwischen zwei oder mehreren Hubs verwendet. Hierfür werden die technischen Primärschlüssel(Surrogatschlüssel) genutzt. Um die Erweiterung technisch einfacher zu machen werden die Beziehungen in einem Data Vault immer als \glqq n:n-Beziehungen\footnote{Eine Entitätsmenge hat mehrere Beziehungen und kann mehrere Beziehungen eingehen}\grqq modelliert. Dadurch muss wegen einer Erweiterung nicht das Datenmodell geändert werden.
	\item {Satelliten}\newline
		Die Satelliten beinhalten alle Attribute, die ein Geschäftskonzept oder eine Beziehung beschreiben. Dies sind beispielsweise Namen und Alter von Kunden. Satelliten enthalten zusätzlich auch Angaben zu Datenquellen und Ladezeit und sind zu Hubs oder Links zugeordnet. Es können zu Hubs oder Links mehrere Satelliten gehören. 
\end{itemize}
Durch dieses Konzept beruht eine Verknüpfung zwischen Entitäten immer auf der Modellierung über Links, die auf Hubs verweisen. 
\cite{Wikipedia:Data_Vault}\cite{DDA:Data_Vault}\cite{DL:Data_Vault}\cite{taod:Data_Vault}

\subsubsection{Data Warehouse nach Inmon}
William Harvey \glqq Bill\grqq Inmon (*20.Juli 1945) gilt als \glqq Vater des Data Warehousing\grqq\cite{Wikipedia:Inmon}. Der Beginn seines Ansatzes ist der des \glqq Enterprise Data Model (EDM)\grqq. Dieses beschreibt die Sicht auf Daten die innerhalb eines gesamten Unternehmens produziert und konsumiert werden. Ein EDM repräsentiert Daten System- oder Anwendungsunabhängig.\cite{TDAN:EDM} Dieses Model identifiziert einen Schlüsselbereich und die Schlüssel-Entitäten mit denen das Unternehmen arbeitet. Bei Bill Inmon handelt es sich bei einem Data Warehouse um \glqq themenorientierte, nichtflüchtige, integrierte, zeitvariante Datenerfassung zur Unterstützung von Managemententscheidungen\grqq. Dies bedeutet:
\begin{itemize}
	\item {Themenorientiert:}\newline Zweck des DW ist nicht die Erfüllung einer dedizierten Aufgabe, sondern die Unterstützung übergreifender Auswertungsmöglichkeiten
	 \item{nicht-flüchtig:}\newline Daten im DW werden i.d.R. nicht mehr geändert
	 \item{integriert:}\newline Daten aus mehreren (inkonsistenten) Datenquellen werden integriert und auf einen Standard standardisiert
	 \item {zeitvariant:} \newline Hitorische Daten, die den Vergleich von Daten über die Zeit ermöglichen und über langen Zeitraum gespeichert werden
\end{itemize}
Daraus wird ein logisches Modell erstellt, welches alle Attribute enthält, die einer  Entität zugeordnet sind.
\begin{quote}
	Beispielsweise können solche Attribute die \underline{Kundennummer}, der KundenName, das Geschlecht und das Alter sein, die sich auf die Entität Kunde beziehen.\cite{ITEW:DataWarehouse}
\end{quote}
Der Ansatz nach Inmon verwendet eine normalisierte Form der Datenmodellierung, um Redundanzen so weit wie möglich zu vermeiden. Daraus ergibt sich eine eindeutige Identifizierung der Anforderungen und eine Vermeidung von Unregelmäßigkeiten wenn Daten aktualisiert werden. Ein weiterer Vorteil dieses Top-Down-Ansatzes ist die Robustheit gegenüber Änderungen und die Erhaltung der Perspektive auf die Daten über einen Data Mart hinweg.
Data Marts beim Inmon-Ansatz in der Regel für jeden Geschäftsbereich seperat erstellt.
\cite{Wikipedia:Inmon}\cite{Astera:Kimmball_Inmon}
\subsubsection{Data Warehouse nach Kimball}
Ralph Kimball beschreibt den Aufbau eines Data Warehouses als Bottom-Up- Architekturdesign. Dieses Design bildet zuerst Data Marts auf Basis der Geschäftsanforderungen, für die das Data Ware-house aufgebaut werden soll. Mittels eines ETL-Tools (Informatica) werden die primären Datenquellen ausgewertet und es werden verschiedene Transformationen durchgeführt, um die Daten in einen Staging-Bereich des relationalen Datenbankservers zu laden. 
Sind die Daten im Staging-Bereich des Data Warehouses umfasst das Laden der Daten in ein Modell, welches bewusst denormalisiert ist. Dieses Modell partitioniert\footnote{lat.: (Ein)teilung von zusammenhängenden aufeinanderfolgender Datenblöcke auf einem Speichermedium} Daten in eine Faktentabelle, die numerische Transaktionsdaten umfasst.
Das Grundelement der Dimension ist das Sternschema. Dieses begrenzt eine Faktentabelle durch mehrere Dimensionen. 
\cite{Astera:Kimmball_Inmon}\cite{ITEW:Patrik_02.05}
\begin{figure}[H]
	\begin{center}
		\includegraphics[scale=0.4]{Images/Star-Schema}
		\caption{Stern-Modellierung nach dem Data Warehouse Ansatz von Bill Inmon}
		\label{fig:Star-Schema}
	\end{center}
\end{figure}
In \autoref{fig:Star-Schema} ist eine Modellierung nach Kimball zu sehen. Es gibt eine zentrale Tabelle \glqq Verkauf\grqq die alle Informationen zum Verkauf enthält. darum herum werden einzelne Attribute zu den Entitäten des Verkaufs in eigenen Tabellen modelliert.

Eine weitere Art des Star-Schemas ist das \glqq Snowflake-Scheme\grqq (Schneflockenschema). Diese Modellierung stellt eine Verbesserung des Star-Schemas dar. Die Faktentabelle bleibt bei diesem Schema unverändert allerdings werden hier die Dimensionen durch Klassifizierung und Normalisierung verfeinert.
\begin{figure}[H]
	\begin{center}
		\includegraphics[scale=0.4]{Images/snowflake}
		\caption{Snowflake-Schema der Datenmodellierung\cite{Wikipedia:Snowflake}}
		\label{fig:snowflake}
	\end{center}
\end{figure}
In \autoref{fig:snowflake} ist die Weiterentwicklung zu sehen. Hier ist erkennbar, dass die Faktentabelle im Zentrum und die Dimensionstabellen darum herum eine Schneeflocke bilden, die sich immer weiter verästeln kann. So ist im Bild die Dimension Ort weiter in mehrere Dimensionen unterteilt. Zum Ort gibt es gewisse weitere Attribute (bspw. Postleitzahl) die in weiteren Dimensionen abgelegt sind. Daher kann man das Snowflake-Schema als in Star-Schema definieren, in dessen Dimensionen weiter Star-Schemen modelliert sind.\cite{tecchanel:snowflake}
Diese dimensionale Modellierung ist eine Abbildung des dimensionalen Würfels der eine Sicht auf einen Sachverhalt aus mehreren Dimensionen ermöglicht.
\begin{figure}[H]
	\begin{center}
		\includegraphics[scale=0.4]{Images/Mehrdim_DAtensicht}
		\caption{Allgemeine Mehrdimensionale Sicht auf Kennzahlen(Faktentabelle) im Data Warehouse}
		\label{fig:Mehrdim}
	\end{center}
\end{figure}
\autoref{fig:Mehrdim} zeigt die allgemeine Mehrdimensionale Sicht auf die Kennzahlen im Data Warehouse. Diese Kennzahlen lassen sich mit verschiedenen Parameter, wie im Beispiel Region, Branche, Zeit, betrachten und es lassen sich dadurch dediziert einzelne oder gebündelte Daten entnehmen. Weiter ist im Würfel auch ein Ansatz der snowflake Modellierung zu sehen, da das Jahr \glqq 2019\grqq in weitere Dimensionen wie zunächst das \glqq Halbjahr\grqq und danach das \glqq Quartal\grqq unterteilt ist.
\cite{Astera:Kimmball_Inmon}\cite{ITEW:Patrik_02.05}\cite{ITEW:DataWarehouse}

\chapter{Extract - Transform - Load: Der ETL Prozess}
Der ETL (Extrct - Transform -Load) Prozess wird im Umfeld des Data Warehousing eingesetzt. Dieser wird dazu genutzt, um Daten aus verschiedenen Datenquellen in eine einheitliche Struktur zu bringen und in einem Data Warehouse abzulegen. 
\begin{itemize}
	\item {Extract - Entnehmen} \newline Im Extract-Prozess werden Daten aus heterogenen Datenbanken extrahiert und ausgelesen
	\item {Transform - Transformieren}\newline Der Transformationsschritt bringt die gelesenen Daten in eine einheitliche Struktur
	\item {Load - Laden}\newline
	Nach der Transformation werden die Daten in das Data Warehouse geschrieben
\end{itemize}
\section{Structured Query Language}
Die Structured Query Language (SQL, Engl.: Strukturierte Abfrage-Sprache) ist eine standardisierte Sprache, zum Abfragen, definieren und bearbeiten von Datenstrukturen. Diese Strukturen und darauf basierende Datenbestände sind in relationalen Datenbanken abgelegt. Daher stellt SQL eine Datenbanksprache dar. 

Die Sprache SQL setzt sich aus 3 verschiedenen Funktionen der zusammen:
\begin{itemize}
	\item {DDL - Data Definition Language:}\newline
	Dient der Definition von Tabellen und anderer Datenstrukturen (Bsp.: Datenbank/ Tabelle erzeugen,...)
	\item {DCL - Data Control Language:}\newline
	Regelt die Kontrolle des Zugriffs (Bsp.: Zugriffsrechte gewähren/entziehen)
	\item {DML - Data Manipulation Language:}\newline
	Befehle für die Manipulation von Daten und der Datenabfrage (Bsp.: Tabellen abfragen/löschen/hinzufügen,...)
\end{itemize}

Allerdings kann SQL auch dazu verwendet werden um Daten zu transformieren. Jedoch handelt es sich hier weniger um einen ETL, als eher um einen ELT-Prozess (Extract-Load-Transform). Da es sich bei SQL um eine Sprache handelt, die nur auf der konkreten Datenbank ausgeführt werden kann, muss zuerst der Schritt der Transformation ausgeführt werden. Hierzu werden Daten direkt in der Datenbank selbst transformiert bevor sie an eine andere Tabelle übergeben werden können. Die Transformation mit SQL bietet den Vorteil, dass diese eine hohe Geschwindigkeit ermöglicht ($\sim 20.000.000$ Datentransformationen in 3 Minuten) und einen hohen Grad an Flexibilität ermöglicht wird.

Durch diesen Grad an Flexibilität wird es ebenfalls möglich SQL-Transformationen auf Basis von gewissen Regeln automatisiert generiert werden können. Dadurch kann der Kunde selbst die Regeln definieren und diese ändern oder anpassen. Dadurch werden SQL-Skripte automatisch generiert und aus den Regeln abgeleitet. 
%\begin{figure}[H]
%	\begin{center}
%		\includegraphics[scale=0.4]{Images/Regeln_Auszug}
%		\caption{Auszug aus Regel-Datei für ZGS\footnote{ZGS:}\cite{Intern}}
%		\label{fig:SQL_Regeln}
%	\end{center}
%\end{figure}
In %\autoref{fig:SQL_Regeln}
ist ein Auszug einer Regel-Datei zu sehen. Hier können Regeln der Transformation geregelt werden und es wird automatisiert ein entsprechender SQL-Skript generiert. 
\lstinputlisting[language=SQL,label={lst:ELT-SQL},caption={Beispiel einer Transformation mittels SQL\cite{SQLORA:Data_Historization}}, numbers=left, stepnumber=1, numberstyle=\tiny,captionpos=b,frame=single, basicstyle=\tiny,breaklines=true]{Listings/ETL.sql}
In \autoref{lst:ETL-SQL}

\cite{ITEW:Alex_02-08}\cite{Wikipedia:SQL}\cite{Hempel:SQL}\cite{Huckert:VL1_2}\cite{SQLORA:Data_Historization}
\section{Das ETL-Werkzeug Informatica}
Informatica Corporation \textsuperscript{\textcopyright} bietet mit Informatica eine Software zur Datenintegration. Mittels dieser ist es möglich einen ETL-Prozess zu entwickeln, der für ein Data Warehousing genutzt werden kann. In gewisser Weise bildet Informatica einen Wrapper\footnote{\glqq Ein Stück Software, welches ein anderes Stück Software umgibt\grqq\cite{Wikipedia:Wrapper}} für und um SQL. Informatica selbst ist hauptsächlich für die Transformation der Daten verantwortlich. SQL bildet den \glqq Zulieferer und Abtransporter\grqq~für die Datensätze. Dies bedeutet, dass mittels SQL die Daten nach Informatica geladen (Extract) und nach der Transformation wieder in eine andere  Datenbank/ein Data Warehouse geladen werden (Load).

Die Transformation selbst kann in Informatica über eine grafische Benutzeroberfläche erstellt werden, wobei der erste Transformationsschritt immer eine Konvertierung in die Datenformate von Informatica darstellt.
\begin{figure}[H]
	\begin{center}
		\includegraphics[scale=0.2]{Images/Infa_Designer}
		\caption{Überblick über den Designer bei Informatica \cite{Intern}}
		\label{fig:Infa_Designer}
	\end{center}
\end{figure}
\autoref{fig:Infa_Designer} zeigt den Überblick über den Informatica Designer. Hier können sog. Mappings gebaut oder über verschiedene Arten Quellen (\glqq Sources\grqq) und Ziele (\glqq Targets\grqq) definiert werden. Diese können sowohl Dateien sein, die lokal auf einem Rechner oder Server vorhanden sind, oder Tabellen, welche selbst in einer Datenbank  vorhanden sind. Weiterhin sind in der linken Spalte alle Ordner zu sehen, die mit dem Repository verknüpft sind. Zu einem Repository können mehrere Ordner mit vielen untergeordneten Quellen und Mappings gehören.
\begin{figure}[H]
	\begin{center}
		\includegraphics[scale=0.6]{Images/Infa_Mapping}
		\caption{Beispiel eines Mappings in Informatica im ETL-Prozess in der Überführung vom LANDING-Layer zum CORE-Layer \cite{Intern}}
		\label{fig:Infa_Mapping}
	\end{center}
\end{figure}
In \autoref{fig:Infa_Mapping} ist ein Beispiel eines Mapping in Informatica zu sehen. Hier wird ein Teil des ETL-Prozesses beschreiben, bei dem Daten aus dem LANDING-Layer in den CORE-Layer überführt werden. Dabei werden, im unteren Strang, die Daten zunächst aus der LANDING-Schicht durch einen sogenannten \glqq Source-Qualifier\grqq~geladen. Da in der LANDING-Schicht alle Daten gelesen werden müssen, sind die vorgegebenen Datenfelder $varchar(100)$-Felder. Dadurch werden alle Zeichen bis zu einer Länge von 100 ohne Fehler eingelesen. Danach müssen alle Daten durch eine \glqq Expression\glqq laufen, damit diese in Datenformate umgewandelt werden können, mit denen Informatica arbeiten kann. Eine kleine Schwierigkeit besteht in den Datumsformaten. Für diese gibt es etliche verschiedene Schreibweisen (YYYY-MM-DD, DD.MM.YYYY, HH12:MI:SS, HH24:MI:SS, um nur ein paar Beispiele hier zu nennen). Diese müssen alle aufgelöst werden, damit Informatica damit arbeiten kann. Dafür kann es sehr von Vorteil sein eigene Funktionen zu definieren, was ebenfalls in Informatica möglich ist. Dazu können im Arbeitsordner über \glqq User-defined Functions\grqq~eigene Funktionen definiert werden.

Der obere Strang liefert die Verarbeitung für die LADE\_ID, die dafür genutzt wird, um eine Lieferung eindeutig zu identifizieren (genaueres zu diesem Beispiel in \autoref{cap:Anwendungsbeispiel} \glqq Die Anwendung des ETL-Prozesses in der Erstellung und Bewirtschaftung des Financial Data Warehouses(FDW)\grqq). Dabei wird hier die Ziel-Tabelle bereits als Quelle definiert, um eine LADE\_ID zu generieren, die sich von der größten bereits vorhandenen LADE\_ID unterscheidet. De facto wird die größte um 1 erhöht. 

Danach werden die Daten mittels eines Joiner-Operators gejoint. Dies bedeutet, dass beide Stränge zusammen in einen geschrieben werden. 
Hier bedeutet das lediglich, dass zu den Daten aus der LANDING-Tabelle die LADE\_ID hinzukommt. Da der Joiner allerdings zwingend eine Bedingung braucht, um korrekt zu arbeiten, müssen zwei \glqq Dummy-Ports\grqq~deklariert werden, die eine triviale Bedingung bilden. Danach werden alle Daten in die Target-Tabelle geschrieben und das Mapping ist abgearbeitet. \\
\\
Allerdings kann Informatica dieses Mapping nicht allein laufen lassen und daher muss es in einen Workflow eingebunden werden. Dieser Workflow kann noch mit einigen anderen Befehlen, wie einen abgesetzten Befehl über die Kommando-Zeile des jeweiligen Betriebssystems, erweitert werden.
\begin{figure}[H]
	\begin{center}
		\includegraphics[scale=0.4]{Images/Infa_Workflow}
		\caption{Informatica-Workflow für den Schritt von Files nach Landing \cite{Intern}}
		\label{fig:Infa_Workflow}
	\end{center}
\end{figure} 
\autoref{fig:Infa_Workflow} zeigt einen solchen Workflow, der den Schritt von \glqq Roh-Dateien\glqq~in die LANDING-Schicht. Dieser ist erweitert durch einen Befehl in der Kommandozeile in Linux (Genaueres über die Bedeutung von BOND und MASTERDATA und den Befehlin \autoref{cap:Anwendungsbeispiel}) 
Dieser Workflow kann dann gestartet werden. Der Status wird dann in Informatica Workflow Monitor angezeigt.
\begin{figure}[H]
	\begin{center}
		\includegraphics[scale=0.6]{Images/Infa_Monitor}
		\caption{Informatica Workflow Monitor \cite{Intern}}
		\label{fig:Infa_Monitor}
	\end{center}
\end{figure}
\autoref{fig:Infa_Monitor} zeigt einen ausgeführten Workflow. Es ist einzeln aufgeschlüsselt, welcher Teil welchen Status hat. So sind in diesem Beispiel alle erfolgreich durchgelaufen. Weiterhin ist es im Workflow Monitor möglich sich eine Log-Datei des Workflows oder der einzelnen Sessions anzeigen zu lassen und eventuelle Error oder Warnings zu überprüfen und zu korregieren. 
\cite{ITEW:Thomas_02-08}\cite{Wikipedia:Informatica}


\chapter{Die Anwendung des ETL-Prozesses in der Erstellung und Bewirtschaftung des Financial Data Warehouses(FDW)}\label{cap:Anwendungsbeispiel}
In diesem Kapitel möchte ich auf ein Anwendungsbeispiel des ETL-Prozesses eingehen. Hierbei handelt es sich um die Erstellung und Bewirtschaftung eines financial Data Warhouses (kurz:FDW) im Umfeld der DZ BANK AG. Dieses FDW wird genutzt, um Geschäftsdaten zu speichern und zur Analyse zu verwenden. Die Quelldaten bestehen aus zwei getrennten Dateien, BOND(engl.: Anleihen) und MASTERDATA. Diese werden immer abhängig von Handelstag des jeweiligen Geschäfts und einem Time\_Stamp geschrieben. 
\begin{figure}[H]
	\begin{center} 
		\includegraphics[width=\linewidth]{Images/FDW_BOND}
		\caption{Beispiel des Datensatzes für BOND\cite{Intern}}
		\label{fig:BOND}
	\end{center}
\end{figure}
\begin{figure}[H]
	\begin{center}
		\includegraphics[width=\linewidth]{Images/FDW_Masterdata}
		\caption{Beispiel des Datensatzes für MASTERDATA\cite{Intern}}
		\label{fig:MASTERDATA}
	\end{center}
\end{figure}

\autoref{fig:MASTERDATA} und \autoref{fig:BOND} zeigen einen Ausschnitt aus den Input-Dateien für das FDW. Diese Daten werden getrennt von einander als csv-Dateien\footnote{comma seperated values, engl., frei: durch Komma getrennte Werte} eingelesen und sollen am Ende in einer Tabelle gespeichert sein. Da es mehrere Handelstage gibt, soll hier das Konzept der Historisierung angewendet werden: Die Handelstage sollen fortlaufend im FDW abgelegt werden. Hier kommt noch eine sogenannte LADE\_ID dazu, die eine Lieferung beziehungsweise ein Geschäft eindeutig zu identifiziert und auch den Handelstag eindeutig zuordnet. Diese ID wird genutzt, um Geschäfte, die an unterschiedlichen Handelstagen eingehen auf ihre Aktualität zu prüfen: Das Geschäft mit der höheren LADE\_ID ist das Neuere.
\section{Datenmodellierung im FDW}
\section{Der ETL-Prozess}\label{sec:ETL_FDW}
Der ETL-Prozess kann hier in diesem Modell in die drei Layer aufteilt werden und wir jeweils etwas anders aufgebaut, um eine gute Strukturierung zu erhalten.
\subsection{Landing-Layer}
Im Landing-Layer werden die Quelldateien als .csv eingelesen. Diese eingelesenen Daten werden dann in die Tabellen LDG\_BOND oder LDG\_MASTERDATA geschrieben. Hierbei sind die erwarteten Datentypen in der Tabelle alle von Typ $varchar(100)$. Dies ermöglicht ein laden aller Daten, auch solcher, die nicht dem endgültigen Datentyp (siehe \autoref{subsec:ETL_COR}) entsprechen. In den LANDING-Tabellen soll allerdings noch keine Historie aufgebaut werden, sondern vor jedem Ladevorgang geleert werden.
\subsection{CORE-Layer}\label{subsec:ETL_COR}
im CORE-Layer werden erste Schritte der Bearbeitung der Daten vorgenommen.

\lstinputlisting[label=Listing:Tab-BONDS,language=SQL,caption={Listing:Tab-BONDS}, numbers=left, stepnumber=1, numberstyle=\tiny,captionpos=b,frame=single, basicstyle=\tiny,breaklines=true]{Listings/ConstraintTableBONDS.sql}

In \autoref{Listing:Tab-BONDS} wird die Tabelle BONDS generiert. Zusammen werden hier auch die \glqq endgültigen Datentypen\grqq~definiert. Diese Datentypen werden gebraucht, um damit Informatica arbeiten kann.  
\begin{figure}[H]
	\begin{center}
		\includegraphics[scale=.6]{Images/Infa_Expression}
		\caption{Umwandlung der Input Daten in die Informatica Datenformate\cite{Intern}}
		\label{fig:Infa_Exp}
	\end{center}
\end{figure}
In \autoref{fig:Infa_Exp} ist eine Funktion zu sehen, die die Inputdaten transformiert. Hier werden alle Datensätze in ein Datenformat umgewandelt, mit dem Informatica arbeiten kann. Diese Transformation ist notwendig, da in der LANDING-Schicht alle Daten in $varchar(100)$-Felder geschrieben werden. Um sinnvoll mit den Daten weiter arbeiten zu können benötigt Informatica spezifischere Datentypen, wie beispielsweise $datetime$ als Datumsformat oder $decimal(s,p)$ für Dezimal mit $s$ Vorkommastellen und $p$ Nachkommastellen.

Bei den Zeilen 11 und 12 kommen \glqq User-defined-functions\grqq~zum Einsatz. Diese ist so konzipiert worden, um in diesem Fall, einen Großteil von gängigen Datumsformaten zu erkennen. Danach werden diese in ein Datumsformat umgeschreiben, sodass Informatica damit arebiten kann. In den Zeilen 13-16 werden mittels der Funktion \glqq TO\_DECIMAL\grqq~alle Zahlenformate erkannt und in eine Dezimalfeld umgewandelt. In den Spalten davor sind gewisse Bedingungen für die jeweilige Zeile angegeben. Beispielsweise ist der Datentyp des jeweiligen Ports, dessen definierte Länge ($Prec$) oder welche Rolle der Port in der Expression einnimmt (Input, Output oder Variable). In Zeile 17 ist ein sogenannter DummyPort angelegt, der in der Expression auf den Wert 1 gesetzt wird. Dies ist notwendig, da der nachgelagerte Join-Operator einen Vergleich von zwei Werten benötigt, um die LADE\_ID zu den Werten , die in \autoref{fig:Infa_Exp} dargestellt sind, zusammenzuführen
\subsection{Aufbereitungs-Layer}
Im Aufbereitungs-Layer werden die Informationen aus den beiden unsprüglichen Tabellen, MASTERDATA und BOND zusammengeführt werden. Basis hierfür sind die jeweiligen Tabellen aus dem CORE-Layer, in dem die Daten schon im richtigen Dateityp vorhanden sind. Als gemeinsamer Schlüssel dient hierbei die \glqq GESCHAEFTS\_ID\_INSTRUMENT\grqq~in Kombination mit dem \glqq HANDELSTAG\grqq~ Sollten zu einem HANDELSTAGmehrere Lieferungen zugeordnet sein, wird jeweils der jüngste Datensatz genommen. Dies bedeutet, der Datensatz mit der größten LADE\_ID zum HANDESTAG. 
Zusätzlich soll ein Feld \glqq NOMINAL\_DELTA\grqq angelegt werden, welches die Differenz der Werte des Nomials aus dem Vortag und desses aus dem aktuellen Handelstag abspeichert.
%einbinden einer Grafik
%\begin{figure}
%    \subfigure[Tabelle LDG_BOND]{\includegraphics[width=0.49\textwidth]{Images/Table_LDG_BONDS}}
%    \subfigure[Tabelle LDG_MASTERDATA]{\includegraphics[width=0.49\textwidth]{Images/Table_LDG_MASTER}}
%    \caption{BOND und MASTERDATA Tabellen im LANDING-Layer mit zugehörigen Datentypen}
%    \label{fig:LDG}
%\end{figure}
%\begin{figure}
%    \subfigure[Tabelle COR_BOND]{\includegraphics[width=0.49\textwidth]{Images/Table_COR_BONDS}}
%    \subfigure[Tabelle COR_MASTERDATA]{\includegraphics[width=0.49\textwidth]{Images/Table_COR_MASTER}}
%    \caption{BOND und MASTERDATA Tabellen im LANDING-Layer mit zugehörigen Datentypen}
%    \label{fig:COR}
%\end{figure}


\chapter{Reflexion}
\section{Beschreibung}
Hier wird das Problem beschrieben
\section{Bewertung}
Hier bewerte ich das Projekt


\printbibliography
\end{document}
